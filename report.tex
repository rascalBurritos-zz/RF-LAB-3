\RequirePackage[OT1]{fontenc} 
\documentclass[journal]{IEEEtran}

% *** CITATION PACKAGES ***
\usepackage[style=ieee]{biblatex} 
\bibliography{example_bib.bib}    %your file created using JabRef

% *** MATH PACKAGES ***
\usepackage{amsmath}

% Table Packages
\usepackage{booktabs}
\usepackage{tabularx}

% *** PDF, URL AND HYPERLINK PACKAGES ***
\usepackage{url}
% correct bad hyphenation here
\hyphenation{op-tical net-works semi-conduc-tor}
\usepackage{graphicx}  %needed to include png, eps figures
\graphicspath{{./images/}}
\usepackage{float}  % used to fix location of images i.e.\begin{figure}[H]

\begin{document}

% paper title
\newcommand{\LabNumber}{\#3}
\newcommand{\LabTitle}{Vector Impedance Measurements 2}

\title{RF Lab Module \LabNumber\ --- \LabTitle}
%\\ \small{Title of the session (you can be creative highlighting your findings)}}

% author names 
\author{Stephen Campbell
    % Student 2 First Name Last Name 
}% <-this % stops a space

% The report headers
\markboth{EE/CE 4202 Electrical and Computer Engineering Laboratory in Circuits. Lab \LabNumber, \today}%do not delete next lines
{Shell \MakeLowercase{\textit{et al.}}: Bare Demo of IEEEtran.cls for IEEE Journals}

% make the title area
\maketitle

% As a general rule, do not put math, special symbols or citations
% in the abstract or keywords.
\begin{abstract}
    In this lab, familiarity with the nanoVNA was gained. Further experience with
    the Agilent vector network analyzer was acquired while several components were
    measured and characterized across a broad frequency range. Specifically, a 50 Ohm
    Load, an antenna, and an attenuator examined.
\end{abstract}

\section{Introduction}

\IEEEPARstart{T}{his} lab required expertise regarding the operation of the
nanoVNA handheld vector network analyzer. The nanoSaver software was used to
save touchstone files (s1p, s2p) so that the measurements could be plotted on a
host computer. After the Agilent VNA and nanoVNA were SOLT calibrated, each
test component was physically connected, examined, measured, and analyzed.

\section{Analysis}
The S-parameters of the devices are plotted with and without calibration on the nanoVNA and Agilent VNA.



\begin{figure}[hp]
    \centering
    \includegraphics[width=0.3\textwidth]{./3db_agilent_smith_chart.png}
    \caption{\label{fig:3db_agilent} \(S_{1,1} \)of 3dB attenuator measured on the Agilent VNA}
\end{figure}

\begin{figure}[hp]
    \centering
    \includegraphics[width=0.3\textwidth]{./3db_nano_smith_chart.png}
    \caption{\label{fig:3db_nano} \(S_{1,1}\) of 3dB attenuator measured on the nanoVNA}
\end{figure}
\begin{figure}[hp]
    \centering
    \includegraphics[width=0.3\textwidth]{./antenna_agilent_smith_chart.png}
    \caption{\label{fig:attenna_agilent} \(S_{1,1}\) of antenna measured on the Agilent VNA}
\end{figure}

\begin{figure}[hp]
    \centering
    \includegraphics[width=0.3\textwidth]{./antenna_agilent_rfplot.png}
    \caption{\label{fig:attenna_agilent_rfplot} \(S_{1,1}\) of antenna measured on the Agilent VNA over frequency}
\end{figure}

\begin{figure}[hp]
    \centering
    \includegraphics[width=0.3\textwidth]{./load_agilent_smith_chart.png}
    \caption{\label{fig:load_agilent} \(S_{1,1}\) of 50 \(\Omega \) measured on the Agilent VNA}
\end{figure}
\begin{figure}[hp]
    \centering
    \includegraphics[width=0.3\textwidth]{./3db_agilent_smith_chart.png}
    \caption{\label{fig:load_nano} \(S_{1,1}\) of 50 \(\Omega \) measured on the nanoVNA}
\end{figure}

\textbf{Compare results between Agilent VNA and NanoVNA}

The Agilent VNA and the NanoVNA had both had accurate results. In some
measurements such as the load measurements, the Agilent was more precise and in
some measurements such as the attenuator measurement the nanoVNA was more
precise. This observation does not take into account the different frequency
ranges that the Agilent and nanoVNA performed. Regardless of this note, the Agilent
and nanoVNA show measurements which correspond to the theoretical values.


\section{Discussion and Summary}

\subsection{Questions to Consider}
\textbf{Make some comments about the difference in calibrating the two VNAs and
    using them for measurements. Accuracy to your knowledge, ease of use, etc.}

\begin{itemize}
    \item \textbf{Calibration}

          The calibration of both instruments is straightforward and relatively easy
          to perform. One note is that the nanoVNA does not require the SOL on the
          second port which reduces the time required to calibrate it.

    \item \textbf{Accuracy and Capabilities}

          To my knowledge, the Agilent is not only more accurate than the
          nanoVNA, but it is also capable of performing accurate vector
          impedance measurements at a higher frequency.

    \item \textbf{Ease of Use}

          Because of the Agilent VNA is significantly heavier and larger than the
          nanoVNA, general handling of the nanoVNA is generally preferred. As far
          as interface easy of use, if both devices were connected and controlled
          by a laptop,  there would not be a major distinction; however, because
          the Agilent requires a flash drive to more files the nanoVNA becomes
          considerably easier to use.

\end{itemize}

% \appendices
% \section{Pre-Lab}
% \section{Extra Photos}

\end{document}